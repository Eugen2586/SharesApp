\documentclass[10pt]{scrartcl}
\usepackage{graphicx}
\usepackage{float}
\usepackage{amsmath}
\usepackage[utf8]{inputenc} 
\usepackage[T1]{fontenc}
\usepackage{lmodern}
\usepackage{amsfonts}   
\usepackage{amssymb}
\usepackage{mathtools}
\usepackage{epstopdf}
\usepackage[shadow]{todonotes}

 
\begin{document}

\title{Dokumentation zum Java Code Camp 21.02.2020}

%Bitte hier Namen Pflegen!
\author{Christian Küllmer, Gianluca Voss}
\date{\today{}, Kassel}
\maketitle
\begin{figure}[H]
	\centering
	\includegraphics[width=0.6\textwidth]{Bilder/Titelblatt/big_logo.png}
\end{figure}
\todo{Bitte pflegt eure Namen auf dieser Seite!}
\newpage
%Inhaltsverzeichnis
\renewcommand{\contentsname}{Inhaltsverzeichnis}
\tableofcontents
\newpage
%Ab hier bitte neuen Stuff von euch pflegen!

\section{Überblick}
%In diesem Abschnitt bitte nur einen Überblick über die Ziele dieses Projekts geben!

In diesem Abschnitt werden die Gedanken, welche bei der Entstehung der App getätigt wurden, erläutert. Man findet hier die Erläuterungen um die einzelnen Gedankengänge beim Entstehen der App nach zu vollziehen und das Gesamprodukt richtig einordnen zu können.

\subsection{Worum geht es in der App Broken Broke Broker}
In der App "Broken Broke Broker" geht es um die Simulation einer Börsenumgebung in der, der Nutzer einzelne Wertpapiere, Währungen oder Kryptowährungen kaufen kann. Auf dem Smartphone wird ein Spiel begonnen indem es um den Handel mit Wertpapieren geht. Die Kursdaten werden dabei aus dem Web gezogen. Die App soll in Java geschrieben werden und die einzelnen Anwendungen sollen möglichst intuitiv von dem Benutzer zu bedienen sein. Die Menüführung soll klar strukturiert und in einem Stil sein, der es möglich macht sich voll auf den Handel mit Devisen und Wertpapieren zu konzentrieren.

\subsubsection{Anforderungen}
Folgende Features sollen mir der oben gewählten Technologie umgesetzt werden:

Bei der oben gewählten Technologie sollen die im Folgenden vorgestellten Hauptfunktionen zur Verfügung gestellt werden.

\begin{enumerate}
	\item 
	Suche nach Aktien, Währungen, Digitalwährungen
	\item
	Anlegung eines Portfolios
	\item
	Übersicht über Depot, Kontostand und Verlauf
	\item
	Darstellung der aktuellen Kurse als Graph
	\item
	Aktien kaufen / verkaufen
	\item
	Transaktionskosten
	\item
	Historie über Käufe / Verkäufe
	\item
	Kaufoption / Verkaufsoption
	\item
	Hintergrundservice für Kauf- / Verkaufsaufträge
	\item
	Spiel zurücksetzen / neustarten
\end{enumerate}


Zu den Hauptfunktionen wurde die Forderung gestellt, dass alle funktionieren müssen. Des Weiteren soll eine Zusatzfunktion implementiert werden und es muss die Stabilität der Applikation gewährleistet sein, darf folglich nicht abstürzen. Eine Dokumentation, sowie eine Präsentation wurde gefordert. Aufgrund der Verbreitung des Coronavirus, wurde hierbei von der Präsentation abgesehen.



\subsection{Motivation}
%In diesem Abschnitt bitte nur beschreiben, wie die Zielvorstellung aussieht.
Die Motivation für uns bei diesem Code Camp die App in Java zu schreiben liegt vor allem darin sich als Team einer unbekannten Herausforderung zu stellen. Eine Herausforderung an deren Ende ein Programm steht, welches Spaß beim Spielen macht und einen Einblick in die Welt des Finanzmarktes gibt. Die App heißt Broken Broke Borker um die Tatsache zu verdeutlichen, dass es um 

\subsection{Herangehensweise}
%Wir die einzelnen Ziele erreicht werden sollten werden wir in diesem Abschnitt hinterlegen!

\section{Funktionalität der Applikation}
%Beschreiben der einzelnen GUI Oberflächen und deren Funktion. 
%Bitte selbstständig die einzlenen Abschnitte mit den Screens ergänzen!


\subsection{Technische Details}

\subsection{Verwendete Programmierbibliotheken}

\begin{enumerate}
	
	\item 
	AnyChart Android

\begin{figure}[H]
	\centering
	\includegraphics[width=0.6\textwidth]{Bilder/BibliothekenLogos/Anychart.png}
\end{figure}

% link: https://www.anychart.com/blog/wp-content/uploads/2016/05/AnyChart_JS_HTML5_Charts_Maps_Stock_Graphs_Dashboards_logo_300x92.png

AnyChart ist eine leichte und robuste JavaScript-Diagrammbibliothek mit hervorragender API, Dokumentation und Unterstützung für Unternehmen.

Es wurde seit 2003 mit einer Hauptidee entwickelt: Es sollte für jeden Entwickler einfach sein, schöne Diagramme in jedes mobile, Desktop- oder Webprodukt zu integrieren. Daher ist AnyChart heute eine Datenvisualisierungsschicht für Tausende großartiger Produkte. Unseres gehört nun auch dazu, wenn auch nur mit der Trial Version.


	\item 
	OkHttpClient

\begin{figure}[H]
	\centering
	\includegraphics[width=0.6\textwidth]{Bilder/BibliothekenLogos/OKHTTP.jpg}
\end{figure}

% link: https://duckduckgo.com/?q=okhttpclient+logo&iax=images&ia=images&iai=https%3A%2F%2Fwww.mkyong.com%2Fwp-content%2Fuploads%2F2019%2F10%2FOkHttp-logo.png

HTTP ist das moderne Anwendungsnetzwerk. So tauschen wir Daten und Medien aus. Durch effizientes Ausführen von HTTP werden Inhalte schneller geladen und Bandbreite gespart.

OkHttp ist ein HTTP-Client, der standardmäßig effizient ist:

Durch die HTTP / 2-Unterstützung können alle Anforderungen an denselben Host einen Socket gemeinsam nutzen.
Das Verbindungspooling reduziert die Anforderungslatenz (wenn HTTP / 2 nicht verfügbar ist).
Transparentes GZIP verkleinert die Downloadgröße.
Durch das Zwischenspeichern von Antworten wird das Netzwerk für wiederholte Anforderungen vollständig vermieden.
OkHttp bleibt bestehen, wenn das Netzwerk Probleme hat: Es wird stillschweigend von häufigen Verbindungsproblemen wiederhergestellt. Wenn der Dienst mehrere IP-Adressen hat, versucht OkHttp, alternative Adressen zu finden, wenn die erste Verbindung fehlschlägt. Dies ist für IPv4 + IPv6 und Dienste erforderlich, die in redundanten Rechenzentren gehostet werden. OkHttp unterstützt moderne TLS-Funktionen (TLS 1.3, ALPN, Pinning von Zertifikaten). Es kann so konfiguriert werden, dass es auf eine breite Konnektivität zurückgreift.

Die Verwendung von OkHttp ist einfach. Die Anforderungs- / Antwort-API wurde mit fließenden Buildern und Unveränderlichkeit entwickelt. Es unterstützt sowohl synchron blockierende calls als auch asynchrone calls mit callback.


	\item
	JSON.simple
	
\begin{figure}[H]
	\centering
	\includegraphics[width=0.6\textwidth]{Bilder/BibliothekenLogos/JSONLogo.png}
\end{figure}

% link: http://aommaster.com/blog/wp-content/uploads/2015/10/JSONLogo.png
	
JSON.simple ist ein einfaches Java-basiertes Toolkit für JSON. Mit JSON.simple kann man JSON-Daten codieren oder decodieren.

	\item
	

\end{enumerate}


\subsection{Navigationsschema zwischen den Bildschirmen}

\subsection{Liste aller Klassen und deren Methoden mit Funktion}

\section{Klassendiagramm}

\subsection{Anbindung an das Datenmodell der API}

\subsection{Darstellung des Datenmodells als UML}

\subsection{Darstellung aller Views und deren Controller als UML}

\section{Überblick über die Umsetzung}
%Hier ist wichtig zu schreiben, was wir meinen Gut gemacht ist und wie dies mit den App zielen conquentiert.

\section{Fazit}
%Hier ist wichtig, dass wir es schaffen können zu zeigen, ob wir mit den erreichten Zielen zufrieden sind.





\end{document}